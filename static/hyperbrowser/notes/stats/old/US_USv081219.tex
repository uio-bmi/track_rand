\documentclass{article}
\usepackage{graphicx}
\usepackage{ifthen,pifont,natbib,amsmath,amssymb,amstext,amsbsy,comment,graphicx,multirow}
\usepackage[latin1]{inputenc}
\usepackage[dvipsnames]{xcolor}

\begin{document}

From: Marit Holden, Lars Holden \\
\hspace{4mm} Date: 19.12.08

\section {US - US }

\subsection {Tracks}

\begin{enumerate}
	\item Track 1: unmarked segments
	\item Track 2: unmarked segments
\end{enumerate}

\subsection{Questions}

\subsubsection{Question 1}

Where in the genome do the segments of track 1 intersect the segments of track 2, more/less/different than expected by chance?



\vspace{5mm}
 Comment:

\begin{itemize}
	\item This question is used to identify regions of the genome (or the part of it under analysis) where 
segments in the two tracks overlap more than expected. 
The test is symmetric in the two tracks. 
	\item Significance is determined by means of p-values. Small p-values identify regions where the segments
in the two tracks overlap more than expected.  P-values are computed as explained below, where the null hypothesis is explained in detail.
	\item The p-values are found by simulation. It is necessary to specify a distribution of the unmarked 
segments. We specify the following distribution. The user specifies bins or if used globally two endpoints. Then we 
assume the length of all segments and all intervals between segments, including interval between first and last
segment and corresponding end point, as fixed. New realizations are simulated by permuting the order of the 
segments and the order of the intervals between segments in the two tracks.   
\end{itemize}


\subsubsection{Question 2}

Where in the genome do the segments of track 1 touch the segments of track 2, more/less/different
 than expected by chance?



\vspace{5mm}
 Comment:

\begin{itemize}
	\item This question is used to identify regions of the genome (or the part of it under analysis) where 
segments in the track 1 touch segments the segments of track 2  more than expected. For each segment in track 2
we find out whether it is overlapping a segment in track 1. The question is then whether more of the 
segments in track 2 are overlapped by segments of track 1 than expected.
The test is not symmetric in the two tracks. 
	\item Significance is determined by means of p-values. Small p-values identify regions where the segments
in the two tracks overlap more than expected.  P-values are computed as explained below, where the null hypothesis is explained in detail.
	\item The p-values are found by simulation. It is necessary to specify a distribution of the unmarked 
segments. We specify the following distribution. The user specifies bins or, if used globally, two endpoints. Then we 
assume the length of all segments and all intervals between segments, including interval between first and last
segment and corresponding end point, as fixed. New realizations are simulated by permuting the order of the 
segments and the order of the intervals between segments in the two tracks.   Here, one may either permute both
tracks or only track 1.
\end{itemize}

\subsubsection{Question 3}

Where in the genome are the segments of track 1  similar to the segments of track 2 with  more/less/different 
frequency than expected by chance?


\vspace{5mm}
 Comment:

\begin{itemize}
	\item This question is used to identify regions of the genome (or the part of it under analysis) where 
segments in the two track are very similar. Similar is defined as follows: Let $S_1$ and $S_2$ be two segments in 
track 1 and track 2 respectively that overlap. Define $S_3$ as the union of $S_1$ and $S_2$ and $l(S)$ as
 the length of a segment $S$. That $S_1$ and $S_2$ are similar is defined as $l(S_1)/l(S_3)>\alpha$ and
 $l(S_2)/l(S_3)>\alpha$
for a constant $\alpha$. The test is then based on the ratio of the segments in the bin that is very similar to a segment 
in the other track.    The test is symmetric in the two tracks. 
	\item Significance is determined by means of p-values. Small p-values identify regions where the segments
in the two tracks overlap more than expected.  P-values are computed as explained below, where the null hypothesis is explained in detail.
	\item The p-values are found by simulation. It is necessary to specify a distribution of the unmarked 
segments. We specify the following distribution. The user specifies bins or if used globally two endpoints. Then we 
assume the length of all segments and all intervals between segments including interval between first and last
segment and corresponding end point as fixed. New realizations are simulated by permuting the order of the 
segments and the order of the intervals between segments in the two tracks.   
\end{itemize}



\subsection {Bins}

The genome (or the areas of the genome under study) are divided into small regions, called bins. The tests are 
performed in each bin.  

\subsection {Hypothesis tested}

\subsubsection {Question 1, overlap}

\begin{itemize}
\item Hypothesis: The overlap is not larger/smaller/different than expected 
	\item Observator for bin $i$: 
$  F_i$ =   length overlap of segments in the two tracks in bin $i$/ 
               total length of segments in the two tracks 
\item Test: $F_i > c_{p,i}$ or  $F_i < d_{p,i}$ or  $  c_{p,i}<F_i < d_{p,i}$  i.e. the overlap 
is significant larger/smaller/different.  The critical values   $  c_{p,i}$ and $ d_{p,i}$ are
 found by simulation and depends on the threshold $p$ and the bin  $i.$  
\end{itemize}

\subsubsection {Question 2, touch}

\begin{itemize}
\item Hypothesis: The frequency of touch of segments from track 1 on segments in track 2 is not larger/smaller/different than expected 
	\item Observator for bin $i$: 
$  G_i $ =  number of  segments track 2 that is touched by segments from  tracks 1 in bin $i$/
               number of segments in track 2
\item Test: $G_i > c_{p,i}$ or  $G_i < d_{p,i}$ or  $  c_{p,i}<G_i < d_{p,i}$  i.e. the frequency of touch 
is significant larger/smaller/different.  The critical values   $  c_{p,i}$ and $ d_{p,i}$ are
 found by simulation and depends on the threshold $p$ and the bin  $i.$  
\end{itemize}

\subsubsection {Question 3, similar segments}

\begin{itemize}
\item Hypothesis: The frequency of similar segments is not larger/smaller/different than expected. 

Note that is test depends on the threshold $\alpha$ defined above.
	\item Observator for bin $i$: 
$  H_i $ =  number of  segments that are similar with segment in the other track in bin $i$/
            number of segments in track 1 and 2
\item Test: $H_i > c_{p,i}$ or  $H_i < d_{p,i}$ or  $  c_{p,i}<H_i < d_{p,i}$  i.e. the frequency of similar segments 
is significant larger/smaller/different.  The critical values   $  c_{p,i}$ and $ d_{p,i}$ are
 found by simulation and depends on the threshold $p$ and the bin  $i.$  
\end{itemize}




\end{document}



